\documentclass{article}
\usepackage{amsmath}

\title{Complex Networks}
\date{\today}
\author{Volker}

\begin{document}
\maketitle

\section{Lecture 1}

\section{Lecture 2}
\label{sec:lecture2}

\begin{itemize}
\item $a_{ij}$ is an element from the adjacency matrix $A$
\item we are looking for all node pairs that are within a cluster
\item the higher $m$, the more modular the network is
\item $X$ is an eigenvector matrix
\item The smallest eigenvalue of the Laplacian is always 0
\item The number of zeros in the eigenvalues of $Q$ tells you how your
  network is connected; if the second eigenvalue is also 0, there are
  two separate parts of the graph
\item If the variance of the degree sequence is large, the eigenvalues
  are also larger
\item if you add more link, the eigenvalue also tend to be larger (or
  stay the same)
\item In a regular graph (where each node has the same degree $d$),
  the largest eigenvalue is just the degree
\item $\beta$ the rate of infection (infection can mean anything;
  e.g. the likelihood of purchase)
\item dynamic reaches a stable state; interesting to see how many
  infections there are in a stable state
\item the epidemic threshold can be approximated by the largest
  eigenvalue
\item $\lambda$ is largest eigenvalue
\item $x$ is principal eigenvector. The principal eigenvector states
  how often a website would be visited in a random walk sceanrio (see
  PageRank)
\item $H_{ij}$: distance from $i$ to $j$ (hop count matrix)
\end{itemize}
\subsection{Network Model - Why do we want to have one?}

Reasons:
\begin{enumerate}
\item Find out which properties lead to this state
\item To simplify the real world and just study the properties of interest
\end{enumerate}

\begin{align}
E[L] = E[\sum_{j>i}a_{ij}] = \sum_{j>i}E[a_{ij}]
\end{align}

Choosing a random node, what is the probability that its degree equals
$k$ in a Erd\"os-Renyi graph?

\begin{align}
Pr[D=k] = {N - 1 \choose k}p^k(1-p)^{N-1-k}
\end{align}

${N - 1 \choose k}$ is the number of possibilities to choose $k$
neighbors.

\begin{align}
E[D] = (N - 1)p
\end{align}

Another equation (TODO: look it up):

\begin{align}
  (1 - p)^{N-1-k} =\\
  e^{\log (1-p)^N-1-k}
\end{align}

\subsection{Wigner's semicircle}

\begin{itemize}
\item histogram or pdf of eigenvalues
\item the largest eigenvaluie creates the small peaks 
\end{itemize}

\subsection{Other stuff}
\begin{itemize}
\item Average clustering coefficient should be large for small-world
  network
\item Given $N, L$ we can construct a Erd\"os-Renyi random graph
  (algorithm: iteratively distribute the links between the nodes)
\item difference between $C_{actual}$ and $C_{random}$: real networks
  actually have the small world property, therefore, their number is
  larger
\item regular graph: each node is connected to $k$ nearest nodes
\item scale-free means that there is no typical value (e.g. power-law
  networks)
\item $\sum_jd_j(t)$ is the total degree of a graph
\end{itemize}

\end{document}
